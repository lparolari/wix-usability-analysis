\documentclass[11pt,a4paper]{article}
\usepackage[utf8]{inputenc}  % .tex chars encoding
\usepackage[T1]{fontenc}     % .pdf chars encoding
\usepackage[italian]{babel}  % translations
\usepackage{amsmath}         % math symbols
\usepackage{amsfonts}        % extended math symbols
\usepackage{amssymb}         % other math symbols
\usepackage{hyperref}        % hyperref
\usepackage{graphicx}        % images

% Macros
% ======
\newcommand*{\wix}{Wix}
\newcommand*{\wixcom}{wix.com}
\newcommand*{\wixcome}{\emph{\wixcom{}}}

% Configs
% =======

% Path relative to the main .tex file 
\graphicspath{ {./img/} }

\author{Luca Parolari}
\title{Web Information Management\\ \large Analisi di usabilità del sito web \wixcom{}}

% Document
% ========

\begin{document}

\maketitle

\clearpage
\tableofcontents

\clearpage

\section{Introduzione}
\label{sec:intro}

Questo documento si prefigge lo scopo di fornire un'analisi di
usabilità del sito web \href{https://wix.com}{\wixcom{}}. Il sito web è
stato analizzato nel periodo di gennaio 2021.

\begin{figure}[h]
  \centering
  \includegraphics[width=0.25\textwidth]{wix-logo}
  \caption{Logo di \wix{}}
  \label{fix:wix-logo}
\end{figure}

\wix{} promuove il suo prodotto principale: un site builder fruibile da
tutti gli utenti sia con che senza conoscenza informatica. I suoi
obiettivi principali sono rendere semplice, immediata, user-friendly e
gratuita la manutenzione del proprio sito web online offrendo
strumenti integrati nel site builder per costruire e modificare blog,
e-commerce e siti vetrina.

\section{Analisi preliminare}
\label{sec:preliminary-analysis}

\section{Homepage}
\label{sec:homepage-analysis}

\subsection{Descrizione generale}
\label{subsec:homepage-description}

\subsection{The Six Ws}
\label{subsec:homepage-the-six-ws}

\section{Pagina interna: Funzionalità}
\label{sec:secondary-page-analysis}

\subsection{Descrizione generale}
\label{subsec:internalpage-description}

\subsection{The Six Ws}
\label{subsec:internalpage-the-six-ws}

\section{Analisi complessiva}
\label{sec:full-analysis}

\subsection{Struttura}
\label{subsec:structure}

\subsection{Navigazione}
\label{subsec:navigation}

\subsection{Pubblicità}
\label{subsec:ads}

\subsection{Immagini}
\label{subsec:images}

\subsection{Registrazione}
\label{subsec:signup}

\section{Altre pagine}
\label{sec:other-pages}

\section{Considerazioni finali}
\label{sec:final-remarks}

\section{Giudizio finale}
\label{sec:final-vote}

\end{document}