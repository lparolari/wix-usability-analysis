\documentclass[11pt,a4paper]{article}
\usepackage[utf8]{inputenc}  % .tex chars encoding
\usepackage[T1]{fontenc}     % .pdf chars encoding
\usepackage[italian]{babel}  % translations
\usepackage{amsmath}         % math symbols
\usepackage{amsfonts}        % extended math symbols
\usepackage{amssymb}         % other math symbols
\usepackage{hyperref}        % hyperref
\usepackage{graphicx}        % images

% Macros
% ======
\newcommand*{\wix}{Wix}
\newcommand*{\wixcom}{wix.com}
\newcommand*{\wixcome}{\emph{\wixcom{}}}

% Configs
% =======

% Path relative to the main .tex file 
\graphicspath{ {./img/} }

\author{Luca Parolari}
\title{Web Information Management\\ \large Analisi di usabilità del sito web \wixcom{}}

% Document
% ========

\begin{document}

\maketitle

\clearpage
\tableofcontents

\clearpage

\section{Introduzione}
\label{sec:intro}

Questo documento si prefigge lo scopo di fornire un'analisi di
usabilità del sito web \href{https://wix.com}{\wixcom{}}. Il sito web è
stato analizzato nel periodo di gennaio 2021.

\begin{figure}[h]
  \centering
  \includegraphics[width=0.25\textwidth]{wix-logo}
  \caption{Logo di \wix{}}
  \label{fix:wix-logo}
\end{figure}

\wix{} promuove il suo prodotto principale: un site builder fruibile da
tutti gli utenti sia con che senza conoscenza informatica. I suoi
obiettivi principali sono rendere semplice, immediata, user-friendly e
gratuita la manutenzione del proprio sito web online offrendo
strumenti integrati nel site builder per costruire e modificare blog,
e-commerce e siti vetrina.

\section{Analisi preliminare}
\label{sec:preliminary-analysis}

\subsection{Contesto}
\label{subsec:context}

\wix{}, in linea generale, è un site builder. Un site builder è uno
strumento software che tipicamente permette la costruzione di siti web
senza (o quasi) scrivere del codice
manualmente.\footnote{\url{https://en.wikipedia.org/wiki/Website_builder}}.
I
site builder si distinguono in due macro categorie: \emph{online} site
builder, dove lo strumento è fornito in cloud ed è gestito
privatamente dal provider che offre il servizio e lo spazio dove
hostare il proprio sito web e \emph{offline} site builder dove invece
lo strumento può essere eseguito sul computer e genera le pagine del
sito che possono poi essere distribuite su un qualunque web hosting.

\wix{} rientra chiaramente nella prima categoria: l'output generato
dal sitebuilder di \wix{} non può essere esportato su altri
provider. Questo rappresenta infatti il core-business di \wix{}: un
servizio di hosting con un grandissimo valore aggiungo, ovvero il site
builder. 

\subsection{Nome del sito}
\label{subsec:site-name}

Il nome \wix{} è ben fatto e segue delle regole ben precise. Esso infatti è

\begin{itemize}
  \item unico;
  \item corto a tal punto che è composto dal minimo numero di lettere
    per creare un nome orecchiabile;
  \item facile da memorizzare e da scrivere, anche se presenta lettere
    come la \emph{w} e la \emph{x} provenienti dall'afabeto
    internazionale e non originariamente presenti nell'afabeto
    italiano. Quest'ultime però sono ormai state sdoganate e tutti ne
    conoscono l'esistenza;
  \item seguito dal dominio ``.com'', molto famoso e facile da
    indovinare anche se non memorizzato;
\end{itemize}

Inoltre, il nome \wix{} risulta correttamente registrato (controllo eseguito su \href{https://www3.wipo.int/branddb/en/}{WIPO
  Global Brand Database}).


\section{Homepage}
\label{sec:homepage-analysis}

\subsection{Descrizione generale}
\label{subsec:homepage-description}

\subsection{The Six Ws}
\label{subsec:homepage-the-six-ws}

\section{Pagina interna: Funzionalità}
\label{sec:secondary-page-analysis}

\subsection{Descrizione generale}
\label{subsec:internalpage-description}

\subsection{The Six Ws}
\label{subsec:internalpage-the-six-ws}

\section{Analisi complessiva}
\label{sec:full-analysis}

\subsection{Struttura}
\label{subsec:structure}

\subsection{Navigazione}
\label{subsec:navigation}

\subsection{Pubblicità}
\label{subsec:ads}

\subsection{Immagini}
\label{subsec:images}

\subsection{Registrazione}
\label{subsec:signup}

\section{Altre pagine}
\label{sec:other-pages}

\section{Considerazioni finali}
\label{sec:final-remarks}

\section{Giudizio finale}
\label{sec:final-vote}

\end{document}
